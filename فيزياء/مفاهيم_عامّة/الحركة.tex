يعتمد علم الحركة على دراسة حركة الأجسام ومختلف الأسباب الّتي تؤدّي إلا تغيّرها أو تعطّلها وتنقسم هذه المعرفة إلى عدّة أقسام منها ما يتعلّق بالخصائص الرّياضيّة للمواقع والمسارات ومنها ما يتعلّق بالأسباب الّتي تؤدّي إلى تغيّر الحركة ومنها ما يتعلّق بماهيّة المكان والزّمان في علاقتهما بالحركة
\begin{section}{مفهوم الحركة}
الحركة مفهوم فطريّ يرتبط عند معظم النّاس بمفهوم مناقض له ألا وهو مفهوم الثّبات أو السّكون فالشّيئ يكون متحرّكا ان لم يكن ثابتا ويكون ثابتا أوساكنا إن لم يكن متحرّكا ولكنّ معنى الثّبات ضبابيّ وغير دقيق فهو عند الكثير من العوامّ مرتبط بالالتصاق بالأرض والتّشبّث بها فالأشجار بهذا المعنى ساكنة وكذلك الجبال والصّخور والبنايات بخلاف الحيوانات والسّيول والسّحب بل والشّمس والقمر وبعض النّجوم

\begin{subsection}{تحديد مفهوم الحركة}
حتّى نتمكّن من تحديد مفهوم الحركة بأكثر قّة يتوجّب علينا تحليل الفروق بين ماهو متحرّك وماهو ثابت وذلك عبر أمثلة تساعدنا على استخلاص الخصائص المميّزة لكلّ من المفهومين

فلنبدأ مثلا بقطعة حجر صغيرة ساكنة على أعلى جبل ولنحاول أن نقارن بينها وبين أخرى مماثلةٍ لها في الحجم والتّكوين تهوي إلى أسفلَ الجبلِ

سيكون أوّل ما يجلب انتباهنا أنّ المسافة الّتي تفصل الحصاة الّتي تهوي إلى الأسفل عنّا وعن غيرنا من الأشياء الثّابتة تتغيّر بمضي الوقت بينما تبقى المسافة الٌتي تفصل الحصاة السّاكنة أعلى الجبل عنّا أو عن أيّ شيئ آخر ساكن قارّة لا تتغيّر ولا تتبدّل فمن هنا نستخلص خاصّيّة بديهيّة لكنّها مهمّة جدّا تخصّ السّكون وهي أنّ
\begin{property}[ثبوت المسافات]
الجسم السّاكن يحافض عبر الوقت على قيمة المسافات الّتي تفصله عن الأجسام السّاكنة الأخرى
\end{property}
ولذلك فإنّنا نقول إنّ الجسم الثّابت يحتلّ مكانا واحدا لا يتبدّل ولا يتغيّر عبر الزّمن ثمّ نعرّف الحركة بأنّها
\begin{definition}[الحركة]
هي تغيّر المكان الّذي يحتلّه الجسم المتحرّك عبر الزّمن
\end{definition}
\end{subsection}
الخاصّيّة الأهمّ في الحركة هي تغيّر المسافات بين الجسم المتحرّك وأجسام تعتبر عادة ساكنة فطالما هذه المسافات تتمدّد أو تتقلّص فإنّ الجسم يعدّ في حالة حركة أمّا إذا ثبتت كلّ المسافات فإنّ الجسم يعدّ في حالة سكون

ولكنّ المعاينة الموضوعيّة تحرج هذا الفهم وتلقي به عرضا لتلفّه الشّكوك ما إن نتذكّر أنّ اللأرض تدور حول الشّمس فإن كانت الأرض تدور حول الشّمس فهي متحرّكة وما تعلّق بمتحرّك فلا يكون ثابتا

ثمّ إنّ القابع في قطار أو في غرفة على ظهر سفينة قد يخيّل إليه أنّ الأشياء ثابتة لا تتحرّك بيدأنّها قد تكون متحرّكة أو ساكنة حسب حالة وسيلة النّقل الّتي تحويها
\begin{subsection}{كيفيّة الحركة}
الآن وقد عرّفنا الحركة سوف نحاول أن نفهم كيف تتجسّد الحركة وكيف يمكن للمُلاحِظِ أن يدرسها دون الخوض في المفاهيم الكمّية وذلك بوصفها بأوصافَ هندسيّةٍ غيرِ حسابيّةٍ وحتّى نتجنّب كلّ المشاكل الّتي تتعلّق بنسبيّة الحركة فسوف نشترط مسبقا أنّ كلّ المُلاحِظين ثابتون على سطح الأرض ممّا يعني أنّهم جميعا ساكنون أحدهم بالنّسبة لآخرين فل نتخيّل أنّ عند أحدهم حصاة وأنّها في يده يراها الآخرون كأنّها نقطة لشدّة صغر كلّ من أبعادها الثّلاثة بالمقارنة مع المسافة الّتي تفصلهم عنها فهنا خاصيّة هندسيّة أولى وهي أنّهُ
\begin{definition}[الجسم النّقطة]
{\large لدراسة حركة جسم ما يمكن تمثيله بنقطة شرط أن تكون أبعاده الثّلاثة صغيرة بالمقارنة مع المسافة الّتي تفصل المُلاحِظ عنه}
\end{definition}
\end{subsection}
\begin{subsection}{نسبيّة الحركة}
ظهر من الأمثلة المذكورة أعلاه أنّ مفهومي السّكون والحركة ليسا بالبديهيّينِ كما يمكن أن نظنّ من الوهلة الأولى وذلك لارتباطها بالملاحِظ وحالته فالمتحرّك بالنّسبة لأحد ما قد يكون ساكنا بالنّسبة للآخر والعكس صحيح أيظا ولكن البعض قد يعترض على هذا القول بدليل أنّ المنطق السّليم لا يجيزُ الشّيئَ وضدّه في آن واحد فكيف يعقل أنّ رجلا واقفا أراه ساكنا وتراه أنت متحرّكا يقطع المسافات وينتقل من مكان لآخر؟

الإجابة قد تكون أسهل ممّا نتوقّع ويكفي أن نتخيّل أنَنا في عربة قطار بها شبابيك مغلقة وقد غفونا قليلا ثمّ استيقظنا فنظرنا إلى ساعتنا ثمّ إلى تذكرة السّفر فتنبّأنا أنّ القطار قد النطلق منذ نصف ساعة وأنّه لابدّ أن يكون قد بلغ أقصى سرعته ولكنّنا لا نشعر بحركته فالمقصورة من النّوع الفاخر جدرانها عازلة للصّوت ونوافذها محكمة الإغلاق لا يدخل منها بصيص من الضّوء والطّريق منبسطة فنشكّ أنّ القطار قد تأخّر عن موعد انطلاقه ولكنّ الفضول يدفعنا للتّأكّد من أنّه لم ينطلق بعد فنحاول فتح النّافذة ولكنّ العربة الفاخرة مجهّزة بنوافذ تفتح بآلة تحكّم عن بعد ونحن لا نجدها فنتشاور ونفكّر في طريقة لنكتشف حالة القطار دون أن نخرج من العربة المكيّفة لشدّة الحرّ نبحث في جيوبنا فنجد قطعا نقديّة فنفكّر أنّنا إن ألقينا بهاإلى الأعل فسوف نراها في حال كان القطار ساكنا ترتفع في حركة عموديّة ثمّ تقع في أيدينا أمّا إذا كان القطار متحرّكا فإنّ القطعة الملقاة في الهواء سوف ترتفع في الهواء عموديّابينما يتقدّم القطار أفقيّا فتسقط القطعة بعيدا عن اليد الّتي ألقتها واللّتي تتحرّك بسرعة القطار لاتّصالها بأرضيّة القطار

\begin{center}\fbox{
\begin{minipage}{\textwidth*3/4}
\begin{small}
لنفترض أنّ القطعة النّقديّة في حالة سكون بالنّسبة للأرض كأن يلقي بها مثلا شخص واقف حذو السّكّة خارج القطار تماما عند مرور المقصورة المذكورة أمامه فعند إلقائها عموديّا فإنّها سترتفع في السّماء وتنخفض سرعتها تدريجيّا حتّى تتوقّف تماما ثمّ تبدأ في الهبوظ حتّى تقع على سطح الأرض في نفس النّقطة الّتي انطلقت منها

أثناء ذلك يكون القطار قد تقدّم المسافة
\begin{amath}
\abox{\Large م}_{\abox{\tiny القطار}} = \abox{\large ز} \times \abox{\Large س}_{\abox{\tiny القطار}}
\end{amath}
بحيث تكون {\Large م} هي المسافة الّتي قطعها القطار الّذي يتحرّك بالسّرعة {\Large س} أثناء المدّة الزّمنيّة {\large ز} الّتي تستغرقها الحركة العموديّة للقطعة النّقديّة
\end{small}
\end{minipage}
}\end{center}

نقوم بإلقاء القطعة فترتفع ثمّ تقع في اليد الّتي ألقتها فنستنتج أنّ القطار لم ينطلق بعد وبينما نحن في أخذ وردّ حول سبب تأخّر القطار إذا بمراقب التّذاكر يمرّ فنسأله عن السّبب فيتعجّب ويخبرنا أنّ القطار قد انطلق في موعده تماما وهو يسير بسرعة 400 كم في السّاعة ثمّ يقوم برفع السّتار لنرى فجأة وكأنّ الأشجار تتطاير أمامنا وكأنّ الأرض قد أصابها الجنون فأخذت تتحرّك أمامنا بسرعة رهيبة ثمّ ننتبه إلى أننّا نحن من بتحرّك مع القطار

هذه التّجربة ليست خياليّة محضة بل يمكن القيام بها من دون صعوبات جمّة وهي سهلة في المفاهيم الّتي ترتكز عليها ولكنّها خطيرة فيما يخصّ النّتائجَ الّتي تقودُ إليها فنحن بصدد اكتشافٍ كبيرٍ غيّرَ وجهَ علمِ الحركةِ بل ونهجَ التّاريخ ككلّ فنحن قد اكتشفنا أنّه لا معنى لمفهومِ حركةِ مطلقةِ بل الحركةُ نسبيّةٌ تتغيّرُ حسب المُلاحِظ
\end{subsection}
\end{section}
